% Add `ngerman` to documentclass for German docs
\documentclass[12pt, a4paper]{article}
\usepackage{a4wide}
\usepackage{setspace}
\usepackage{csquotes}
\usepackage[utf8]{inputenc}

\usepackage{url}
\usepackage[hidelinks]{hyperref}
\usepackage{minted}
\usemintedstyle{perldoc}

% inline code
\newcommand{\code}[1]{\texttt{#1}}

% Uncomment for German
%\usepackage[ngerman]{babel}

% For generating template dummy text
\usepackage{lipsum}

\usepackage{myColors}
\usepackage{myFooter}
\usepackage{myTitle}

% Libraries outside of template
\usepackage[T1]{fontenc}
\usepackage{upquote}
\AtBeginDocument{%
    \def\PYZsq{\textquotesingle}%
}


\project{CS 432 Web Science}
\author{Derek Goddeau}
\title{Assignment Three}
\supervisor{Michael L. Nelson}

\doublespace
\pagestyle{hacker}

\begin{document}
\maketitle

\newpage



%%%%%%%%%%%%%%%%%%%%%
% Download URI HTML %
%%%%%%%%%%%%%%%%%%%%%
\section{Download the HTML for the 1000 URIs}

To download the HTML for the URIs the bash script \code{get\_html.sh}
is used. The \code{fetch()} function does all the work, downloading the
HTML using \code{wget} and creating a \code{SHA-1} hash of the URI
to store locally.

\begin{minipage}{\linewidth} % prevent splitting between pages
\vspace{2em}
\begin{minted}[fontfamily=tt]{bash}
fetch() {
    while read uri; do
        local hash=$(echo -n "$uri" | sha1sum | cut -d ' ' -f 1)
        local hash+=".html"
        wget -O data/raw_html/"$hash" "$uri"
        if [[ "$?" != 0 ]]; then
            echo >&2 '[*] Error downloading file'
            FAILURES=$(expr FAILURES + 1)
        else
            echo >&2 '[*] Success'
        fi
    done < "$FILE"
}
\end{minted}
\end{minipage}


%%%%%%%%%
% TFIDF %
%%%%%%%%%
\section{Calculate TFIDF}

\begin{minipage}{\linewidth} % prevent splitting between pages
\vspace{2em}
\begin{minted}[fontfamily=tt]{python}
def get_num_mementos(link):
    url = 'http://memgator.cs.odu.edu/timemap/json/http://' + link
    try:
        mementos = requests.get(url).json()
    except ValueError as e:
        print("No memento for URL: {}".format(link))
        return 0
    num_mementos = len(mementos['mementos']['list'])
    return num_mementos

link_mementos = []
for link in final_links:
    link_mementos.append((link, get_num_mementos(link)))
\end{minted}
\end{minipage}

\begin{minipage}{\linewidth} % prevent splitting between pages
\vspace{2em}
\begin{minted}[fontfamily=tt]{r}
%%R -i data

    library(plotly);

    p <- plot_ly(x = data, type = "histogram")

    embed_notebook(p)
    htmlwidgets::saveWidget(as.widget(p), "histogram.html")
\end{minted}
\vspace{2em}
\end{minipage}


%%%%%%%%%%%%%%%%%%%%
% Rank by PageRank %
%%%%%%%%%%%%%%%%%%%%
\section{Rank by PageRank}

\end{document}
