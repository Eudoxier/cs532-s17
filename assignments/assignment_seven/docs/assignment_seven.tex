% Add `ngerman` to documentclass for German docs
\documentclass[12pt, a4paper]{article}
\usepackage{a4wide}
\usepackage{setspace}
\usepackage{csquotes}
\usepackage[utf8]{inputenc}

\usepackage{url}
\usepackage[hidelinks]{hyperref}
\usepackage{minted}
\usemintedstyle{perldoc}

% inline code
\newcommand{\code}[1]{\texttt{#1}}

% Uncomment for German
%\usepackage[ngerman]{babel}

% For generating template dummy text
\usepackage{lipsum}

\usepackage{myColors}
\usepackage{myFooter}
\usepackage{myTitle}

% Libraries outside of template
\usepackage[T1]{fontenc}
\usepackage{upquote}
\AtBeginDocument{%
    \def\PYZsq{\textquotesingle}%
}

\usepackage{amsmath}

%%%%%%%%%%%%%%%%%%%%%%%%%%%%%%%%%%%%%%%%%%%%%%%%%%%%%%%%%%%%%%%%%%%%%%%

\project{CS 432 Web Science}
\author{Derek Goddeau}
\title{Assignment Seven}
\supervisor{Michael L. Nelson}

\doublespace
\pagestyle{hacker}

\begin{document}
\maketitle

\newpage



%%%%%%%%%%%%
% Facebook %
%%%%%%%%%%%%
\section{Find a substitute you}

Because of the structure of the data I chose to do the entire assignment in R. It is much easier to deal with so many data frames with R as long as there is no need for anything other than data manipulation. After reading each dataset into a data frame with the same name I took a subset of \code{u.user} for users similar to me.

\begin{minipage}{\linewidth} % prevent splitting between pages
\vspace{4em}
\begin{minted}[fontfamily=tt]{R}
df <- u.user[ u.user$age == 29
             & u.user$gender == 'M'
             & u.user$occupation == 'programmer', ]
\end{minted}
\vspace{4em}
\end{minipage}

\noindent
This only resulted in two users.

\begin{minipage}{\linewidth} % prevent splitting between pages
\vspace{4em}
\centering
\begin{tabular}{|l|r|r|l|l|l|}
    \hline
      & user.id & age & gender & occupation & zip.code\\
    \hline
    45 & 45 & 29 & M & programmer & 50233\\
    \hline
    222 & 222 & 29 & M & programmer & 27502\\
    \hline
\end{tabular}
\vspace{4em}
\end{minipage}

\noindent
But another try with \code{u.user\$occupation == \textquotesingle scientist\textquotesingle} got one more hit.

\begin{minipage}{\linewidth} % prevent splitting between pages
\vspace{4em}
\centering
\begin{tabular}{|l|r|r|l|l|l|}
    \hline
      & user.id & age & gender & occupation & zip.code\\
    \hline
    483 & 483 & 29 & M & scientist & 43212\\
    \hline
\end{tabular}
\vspace{4em}
\end{minipage}


Then for each of the three resulting users I created a list of all of their ratings by subsetting the \code{u.data} data frame and then matching each item ID to each movie ID and keeping the movie names.

\begin{minipage}{\linewidth} % prevent splitting between pages
\vspace{2em}
\begin{minted}[fontfamily=tt]{R}
u45.data <- u.data[ u.data$user.id == 45, ][c("item.id", "rating")]
u45.data$item.id <- u.item$movie.title[match(u45.data$item.id, 
                                             u.item$movie.id
                                            )
                                      ]
\end{minted}
\vspace{2em}
\end{minipage}

\noindent
This resulted in a data frame as like the following table.

\begin{minipage}{\linewidth} % prevent splitting between pages
\vspace{2em}
\centering
\begin{tabular}{|l|r|r|l|l|l|}
        \hline
        item.id & rating\\
        \hline
        Birdcage, The (1996) & 4\\
        \hline
        Mystery Science Theater 3000: The Movie (1996) & 5\\
        \hline
        Twister (1996) & 4\\
        \hline
        Dragonheart (1996) & 3\\
        \hline
        Godfather, The (1972) & 5\\
        \hline
        Independence Day (ID4) (1996) & 4\\
        \hline
\end{tabular}
\vspace{2em}
\end{minipage}

From here I counted each good rate and bad rate for each of the three users. I generated a score for each by subtracting the total number of bad rates from the total number of good rates. None had a good score, but User 45 and User 222 were close and User 222 got disqualified by giving The Fifth Element a bad rating. The chosen substitute me, User 45, ended up with a score of $-3$ with three more ratings I considered bad than good. So not very representative, the biggest outliers were rating Twister a $4$ and Willy Wonka and the Chocolate Factory $2$.

\section{Find top and bottom five correlated users to you}

To find correlations in the data I used the \code{cor()} function throughout. First I gathered a list of data frames for each user's ratings.

\begin{minipage}{\linewidth} % prevent splitting between pages
\vspace{2em}
\begin{minted}[fontfamily=tt]{R}
user.list <- list()
for(n in 1:dim(u.user)) {
        user.list[[n]] <- u.data[ u.data$user.id == n, ][c("item.id", "rating")]
}
\end{minted}
\vspace{2em}
\end{minipage}

Then I define the \code{cordf} function which correlates two dataframes passed to it and use \code{sapply()} over the list to compare each user to the subsitute me.

\begin{minipage}{\linewidth} % prevent splitting between pages
\vspace{2em}
\begin{minted}[fontfamily=tt]{R}
cordf <- function(df.one, df.two) {
    df.one <- df.one[ df.one$item.id %in% df.two$item.id, ]
    df.two <- df.two[ df.two$item.id %in% df.one$item.id, ]
    df.one <- df.one[order(df.one[,1]), ]
    df.two <- df.two[order(df.two[,1]), ]

    # pearson is the default
    cor(df.one$rating, df.two$rating)
}

cors <- list()
cors <- sapply(user.list, cordf, df.one=sub.me)
\end{minted}
\vspace{2em}
\end{minipage}

\newpage
I use the same exact pattern to create a vector of the number of items each correlation calculation was performed with and merge them into a single data frame, and removing substitute me from the data. The \code{cor()} function requires at least three data points to calculate the correlation but there was a large number of ties for most and least correlated, and with only three data points it is possible that some were lucky collisions. I filtered the data frame removing all users that did not have more than $5$ movie ratings in common with substitute me.

\begin{minipage}{\linewidth} % prevent splitting between pages
\vspace{2em}
\begin{minted}[fontfamily=tt]{R}
cor.data.filtered <- cor.data[cor.data$incommon > 5, ]
\end{minted}
\vspace{2em}
\end{minipage}

\noindent
From here the top five correlated users can easily be found using \code{order()}.

\begin{minipage}{\linewidth} % prevent splitting between pages
\vspace{2em}
\begin{minted}[fontfamily=tt]{R}
head(cor.data.filtered[ order(cor.data.filtered[,1], decreasing=TRUE), ])
\end{minted}
\vspace{2em}
\end{minipage}

\begin{minipage}{\linewidth} % prevent splitting between pages
\vspace{2em}
\centering
\begin{tabular}{|l|r|r|l|l|l|}
        \hline
          & correlation & incommon\\
        \hline
        728 & 1.0000000 & 6\\
        \hline
        210 & 0.9338430 & 12\\
        \hline
        871 & 0.9284767 & 6\\
        \hline
        409 & 0.8931977 & 6\\
        \hline
        71 & 0.8876254 & 7\\
        \hline
        610 & 0.8750000 & 6\\
        \hline
\end{tabular}
\vspace{2em}
\end{minipage}


\newpage
Also, the least five correlated.

\begin{minipage}{\linewidth} % prevent splitting between pages
\vspace{2em}
\begin{minted}[fontfamily=tt]{R}
head(cor.data.filtered[ order(cor.data.filtered[,1], decreasing=FALSE), ])
\end{minted}
\vspace{2em}
\end{minipage}

\begin{minipage}{\linewidth} % prevent splitting between pages
\vspace{2em}
\centering
\begin{tabular}{|l|r|r|l|l|l|}
        \hline
          & correlation & incommon\\
        \hline
        647 & -0.8783101 & 6\\
        \hline
        196 & -0.8212037 & 8\\
        \hline
        217 & -0.7083333 & 7\\
        \hline
        677 & -0.5519432 & 8\\
        \hline
        199 & -0.5510141 & 6\\
        \hline
        636 & -0.5477226 & 7\\
        \hline
\end{tabular}
\vspace{2em}
\end{minipage}


\newpage
\section{Get your top and bottom 5 recommendations}

To get the top and bottom five recommendations I start by creating a correlation matrix for all of the movies. 

\begin{minipage}{\linewidth} % prevent splitting between pages
\vspace{2em}
\begin{minted}[fontfamily=tt]{R}
# Slice only the genre data
item.data <- u.item[c(-1:-5)]
rownames <- rownames(item.data)

# Invert and replace rownames
t.item.data <- as.data.frame(t(item.data))
colnames(t.item.data) <- rownames
item.cors <- cor(t.item.data)
\end{minted}
\vspace{2em}
\end{minipage}

This results in a symmetric matrix so it is true that,

\begin{align*}
    A\:&=\:A^{\mathrm{T}}\\
    A_{ij}\:&=\:A_{ji}
\end{align*}

I then create $6$ helper functions which consist of \code{top.5.items}, \code{top.5.users}, \code{top.5.user.items}, and their bottom counterparts. each one returns only a numeric vector of item or user IDs.

\newpage
The helper functions are used in the \code{get.ratings} function to return a data frame with item IDs and a score. To calculate the score first the top five most correlated users to substitue me are found using the \code{top.5.users} function and each of their top five movies are found. Then I get the five most correlated movies to each of their favorite movies. The item IDs for each is added to vector including duplicates.

\begin{minipage}{\linewidth} % prevent splitting between pages
\vspace{2em}
\begin{minted}[fontfamily=tt]{R}
# Top 5's favorites
top.users <- top.5.user()
item.id <- unlist(lapply(top.users, top.5.user.items, target.id=u.id))

# Get movies similar to my top 5 users top 5
item.id <- append(item.id,
                      unlist(lapply(unique(item.id), top.5.items, 
                                    u.id=45, 
                                    i.cors=item.cors
                                   )
                            )
                     )
\end{minted}
\vspace{2em}
\end{minipage}
    
Then I get my top five but only add items similar to the vector. Last the items in the vector are counted and the results returned as a data frame using \code{as.data.frame(table(item.id))}.

\newpage
Top 5 movie recommendations:

\begin{minipage}{\linewidth} % prevent splitting between pages
\vspace{2em}
\centering
\begin{tabular}{|l|r|r|l|l|l|}
    \hline
      & item.id & Freq.x & Freq.y & total & movie.title\\
    \hline
    1 & 159 & 3 & 0 & 3 & Basic Instinct (1992)\\
    \hline
    2 & 204 & 3 & 0 & 3 & Back to the Future (1985)\\
    \hline
    3 & 291 & 3 & 0 & 3 & Absolute Power (1997)\\
    \hline
    4 & 479 & 3 & 0 & 3 & Vertigo (1958)\\
    \hline
    5 & 56 & 3 & 0 & 3 & Pulp Fiction (1994)\\
    \hline
\end{tabular}
\vspace{2em}
\end{minipage}

Bottom 5 movie recommendations:

\begin{minipage}{\linewidth} % prevent splitting between pages
\vspace{2em}
\centering
\begin{tabular}{|l|r|r|l|l|l|}
    \hline
      & item.id & Freq.x & Freq.y & total & movie.title\\
    \hline
    1 & 26 & 3 & 8 & -5 & Brothers McMullen, The (1995)\\
    \hline
    2 & 40 & 3 & 8 & -5 & To Wong Foo, Thanks for Everything! (1995)\\
    \hline
    3 & 41 & 3 & 8 & -5 & Billy Madison (1995)\\
    \hline
    4 & 42 & 3 & 8 & -5 & Clerks (1994)\\
    \hline
    5 & 67 & 2 & 6 & -4 & Ace Ventura: Pet Detective (1994)\\
    \hline
\end{tabular}
\vspace{2em}
\end{minipage}



\newpage
\section{Get films related to favorite and least favorite}

My favorite movie on the list is \emph{The Fifth Elemet} and the most correlated movies to it are:

\begin{minipage}{\linewidth} % prevent splitting between pages
\vspace{2em}
\centering
\begin{tabular}{|l|r|r|l|l|l|}
        \hline
        correlation & title\\
        \hline
        1.00000000 & Timecop\\
        \hline
        1.00000000 & No Escape\\
        \hline
        1.00000000 & Highlander III\\
        \hline
        1.00000000 & Barb Wire\\
        \hline
        1.00000000 & Demolition Man\\
        \hline
\end{tabular}
\vspace{2em}
\end{minipage}

\noindent
I would say the only one totally out of place is \emph{Barb Wire}. The bottom 5 are:

\begin{minipage}{\linewidth} % prevent splitting between pages
\vspace{2em}
\centering
\begin{tabular}{|l|r|r|l|l|l|}
        \hline
        correlation & title\\
        \hline
        -0.204980 & Space Jam\\
        \hline
        -0.204980 & Hercules\\
        \hline
        -0.177123 & Legends of the Fall\\
        \hline
        -0.177123 & Professional, The\\
        \hline
        -0.177123 & Bound\\
        \hline
\end{tabular}
\vspace{2em}
\end{minipage}

Besides \emph{Space Jam} this is pretty accurate.

\newpage
\noindent
My least favorite film from the list is \emph{Mystery Science Theater 3000}:

\begin{minipage}{\linewidth} % prevent splitting between pages
\vspace{2em}
\centering
\begin{tabular}{|l|r|r|l|l|l|}
        \hline
        correlation & title\\
        \hline
        1.00000000 & Sleeper\\
        \hline
        1.00000000 & Delicatessen\\
        \hline
        1.00000000 & Back to the Future\\
        \hline
        1.00000000 & Coneheads\\
        \hline
        1.00000000 & Junior\\
        \hline
\end{tabular}
\vspace{2em}
\end{minipage}

\noindent
I have no idea about \emph{Sleeper}, \emph{Delicatessen}, or \emph{Junior}, and \emph{Back to the Future} is definetely not similar. \emph{Coneheads} is spot on but that shows how this isn't always a good metric because I do like that movie.

\begin{minipage}{\linewidth} % prevent splitting between pages
\vspace{2em}
\centering
\begin{tabular}{|l|r|r|l|l|l|}
        \hline
        correlation & title\\
        \hline
        -0.204980 & Diva\\
        \hline
        -0.204980 & Pagemaster, The\\
        \hline
        -0.177123 & Legends of the Fall\\
        \hline
        -0.177123 & Professional, The\\
        \hline
        -0.177123 & Bound\\
        \hline
\end{tabular}
\vspace{2em}
\end{minipage}

I have no idea about \emph{Diva} or \emph{The Pagemaster}, but they do not sound like they are similar.

\end{document}
